%%%%%%%%%%%%%%%%%%%%%%%
%%%%%%%% SETUP %%%%%%%%
%%%%%%%%%%%%%%%%%%%%%%%

\documentclass[11pt]{article}

% --- Multi-file ---
\usepackage[subpreambles=true, sort=true]{standalone}
\usepackage{import}

% --- Bibtex ---
\usepackage[ style=ieee ]{biblatex}
\addbibresource{references.bib}

% --- Pictures ---
\usepackage{graphicx} % Pictures
\usepackage{tikz}

% --- Page Layout ---
\usepackage{tabularx} % Advanced tables
\usepackage{setspace} % Line spacing
\usepackage[hidelinks]{hyperref} % Hyperlinks

% Margins
\usepackage[
	left=2.0cm,
	right=2.0cm,
	top=2.0cm,
	bottom=2.5cm,
]{geometry}

% Page text columns
\usepackage{multicol}
\setlength{\columnsep}{0.9cm}

% Other setup
\parindent 0pt
\parskip 3mm



%%%%%%%%%%%%%%%%%%%%%%%%
%%%%%%% DOCUMENT %%%%%%%
%%%%%%%%%%%%%%%%%%%%%%%%

\begin{document}
\raggedcolumns

% ----- TITLE PAGE -----
\import{}{title_page}


\newgeometry{
	left=2.0cm,
	right=2.0cm,
	top=2.0cm,
	bottom=2.5cm,
}
% ----- MAIN DOCUMENT -----
\begin{multicols}{2}
	\begin{spacing}{1.125}
		\tableofcontents

		\section{Introduction}
		Cryptography secures virtually all digital communication today, yet its
		core concepts can seem abstract without hands‑on exploration. In this
		project, we present a unified toolkit that brings three cryptographic
		techniques to life: \textit{Frequency Analysis}, \textit{RSA}, and the
		\textit{One‑Time-Pad}. We will talk about how each tool works and why they
		are useful.


		% ----- Tool Headers -----
		\section{Frequency Analysis}
		The frequency analysis tool developed in this project is designed to
		deconstruct any given text by counting the occurrences of each character or
		symbol. Essentially, the tool converts raw text into a statistical
		distribution, revealing hidden patterns within the data. By iterating over
		every character in a string and maintaining a count of its occurrences
		using Python’s dictionary data structure, the tool generates a clear
		frequency profile. This profile is particularly valuable in the context of
		classic substitution ciphers, as certain letters—such as \textit{'E'} in
		English—tend to appear more frequently than others. In a ciphertext, such
		frequency patterns can guide cryptanalysts in making informed guesses about
		the correspondence between the encrypted characters and common letters in
		the target language, thereby aiding the cryptographic attack process.

		In any natural language, some letters and symbols occur more frequently
		than others. For instance, English texts typically feature \textit{'E'} as
		the most common letter, followed by letters like \textit{'T'},
		\textit{'A'}, and \textit{'O'}. A transposition cipher which scrambles
		all the letters in a plaintext will perfectly preserve these
		natural frequencies. This means that even though the letters have been
		replaced, the relative proportions of each character remain fully
		intact. By carefully counting how many times each symbol appears in the
		ciphertext, it is possible to generate a frequency profile—a kind of
		statistical fingerprint of the text. This profile can then be compared to
		the known frequency distribution of the target language. This makes it
		easier to decipher the original ciphertext.

		\section{One-Time Pad}
		This OTP tool, implemented in Python, provides two complementary interfaces
		— \textit{textPad} for alphanumeric messages and \textit{digitalPad} for
		binary/hexadecimal data — anchored by the \textit{OTPmain} menu. In text
		mode, the program sanitizes user input to include only alphanumeric
		characters, then either accepts a user‑provided pad or generates a truly
		random pad of equal length (repeating it if necessary). Encryption maps
		each character to its index in lettercodes, adds the corresponding pad
		index modulo 36, and converts the result back to a character; decryption
		subtracts the pad index instead of adding it. In digital mode, the tool
		auto‑detects whether the input is binary, octal, decimal, or hexadecimal,
		parses it into an integer, and similarly accepts or generates a random key
		of matching bit‑length. It then performs a bitwise XOR between the message
		integer and the key, outputting ciphertext and pad in both binary and
		hexadecimal formats. Both modes loop to allow repeated operations until the
		user chooses to exit—achieving perfect secrecy across both text and digital
		data.

		The one‑time pad achieves perfect secrecy by combining plaintext with a
		truly random, single‑use key at least as long as the message, ensuring that
		the resulting ciphertext is statistically independent of — and thus reveals
		no information about — the original text. By providing both a
		modular‑arithmetic interface for alphanumeric messages and a bitwise‑XOR
		interface for binary/hex data, this tool vividly demonstrates the three
		critical requirements for OTP security—genuinely random key generation,
		strict one‑time use, and secure key management—offering a hands‑on
		exploration of information‑theoretic security and one-time-pad remains the
		only cryptosystem to have perfect secrecy.

		\section{RSA}
		This RSA tool, implemented in Zig, parses and formats both public and
		private keys based on industry standards (RFC 4253 for public keys and RFC
		8017 for private keys). It converts text messages into big integers and
		then uses modular exponentiation to perform encryption and decryption. Key
		functions include decoding base-64 PEM formatted keys, extracting
		components like exponents and moduli, and efficiently computing
		large-number arithmetic (such as computing multiplicative inverses via the
		extended Euclidean algorithm), all of which work together to securely
		transform data.

		RSA is very valuable for cryptography because it facilitates secure
		communication through the use of asymmetric key pairs — one key for
		encryption and a different one for decryption. The method relies on complex
		mathematical operations, particularly the difficulty of factoring large
		prime numbers, making it practically infeasible to reverse without the
		corresponding private key. This inherent challenge ensures that only those
		with the proper private key can decrypt and access the data. These
		properties make RSA great for protecting data transmissions, verifying
		digital signatures, exchanging keys for other non-public-key cryptosystems,
		and supporting secure authentication protocols across various applications.



		% --- Resources ---
		\section{Resources}
		\subsection{Frequency Analysis}
		\begin{itemize}
			\item \citetitle{101computing_frequency_analysis_2019} \cite{101computing_frequency_analysis_2019}
		\end{itemize}
		\subsection{One-Time-Pad}
		\begin{itemize}
			\item \citetitle{rublon_onetime_password_vs_pad_2024} \cite{rublon_onetime_password_vs_pad_2024}
		\end{itemize}
		\subsection{RSA}
		\begin{itemize}
			\item \textit{\citefield{rfc4253}{howpublished}:} \citetitle{rfc4253}\cite{rfc4253}
			\item \textit{\citefield{rfc8017}{howpublished}:} \citetitle{rfc8017} \cite{rfc8017}
			\item \citetitle{wiki:ASN.1} \cite{wiki:ASN.1}
			\item \citetitle{wiki:X.690} \cite{wiki:X.690}
			\item \citetitle{wiki:PKCS_1} \cite{wiki:PKCS_1}
			\item \citetitle{wiki:RSA_cryptosystem} \cite{wiki:RSA_cryptosystem}
			\item \citetitle{rsa_encryption_brilliant} \cite{rsa_encryption_brilliant}
		\end{itemize}


		% --- Contributions ---
		\section{Contributions}
		\subsection{Clark, James}
		\begin{itemize}
			\item Frequency Analysis
			\item GUI/App
		\end{itemize}
		\subsection{Davies, Ethan}
		\begin{itemize}
			\item Writing Portion
		\end{itemize}
		\subsection{Owen, Gwilym}
		\begin{itemize}
			\item One-Time-Pad
		\end{itemize}
		\subsection{Shiells Thomas, Isaac}
		\begin{itemize}
			\item RSA
			\item One-Time-Pad (Refactor)
			\item Code Cleanup
			\item Latex
			\item File Organization
		\end{itemize}

		% --- References ---
		\printbibliography

	\end{spacing}
\end{multicols}

\end{document}
