\documentclass{article}
\usepackage[left=1in, right=1in, top=1.5in, bottom=1.5in]{geometry}

\title{CPSC 329 Project Proposal}
\author{\textbf{Group 5}\footnote{The fifth group member has not responded to any attempt at communication, and so was not included in this proposal.} \\ Davies, Ethan, 30106437 \\ Owen, Gwilym, 30188047\\ Shiells Thomas, Isaac, 30176473 \\ Clark, James, 30213187}

\date{6 March, 2025}


\begin{document}
\maketitle
\section*{Project Title}
Our project title is \textbf{the use cases of firewalls in cybersecurity}.

\section{Project Description}
Firewalls are a key component of cybersecurity. They are designed to protect computer networks by filtering out harmful traffic while allowing legitimate connections. While widely used, firewalls are not a one-size-fits-all solution. Our project will explore how firewalls function in cybersecurity, particularly within medium to large-sized businesses. We aim to determine when firewalls are an effective security measure, when they may be insufficient, and what additional security strategies businesses should implement alongside them.	

\section{Research statement}
This project will analyze the effectiveness of firewalls in corporate cybersecurity. We will examine cases where firewalls provide strong protection, as well as scenarios where they may fall short. By reviewing academic research and real-world case studies, we seek to establish the situations in which firewalls are the most suitable defense mechanism and when alternative or complementary security measures should be considered.

\section{Intended goals}
The intended goals of our project are to identify scenarios in which firewalls are essential, explore cases in which they may not be the best solution, and offer recommendations to businesses on how to enhance their cybersecurity strategies. Additionally, we will assess best practices for firewall implementation, ensuring they are used as part of a comprehensive security framework.

\section{Research significance}
Overreliance on firewalls can create security vulnerabilities, particularly as cyber threats evolve. Businesses that depend solely on firewalls without integrating additional security measures may expose themselves to risks. Our research will highlight firewalls' strengths and limitations, providing businesses with the insights they need to make informed cybersecurity decisions. Understanding the role of firewalls within a broader security strategy is crucial for organizations aiming to build more resilient defense systems.

\section{Scope}
The scope of this project will focus on medium to large-sized businesses and their approach to firewall deployment. We will examine industry standards, compliance requirements, and real-world implementations to better understand how companies integrate firewalls into their cybersecurity infrastructure. Furthermore, we will compare traditional firewall configurations with modern, adaptive firewall technologies to evaluate their effectiveness in different business environments.

\section{Intended approach}
To conduct this research, each team member will focus on a specific aspect of firewall use. We will analyze academic literature, industry reports, and case studies to develop a well-rounded perspective on the advantages and limitations of firewalls in corporate cybersecurity. Our final report will consolidate these findings and provide actionable recommendations for businesses on how to optimize their firewall strategies. By following this approach, we aim to offer valuable insights that help companies strengthen their cybersecurity posture and make more informed security decisions.



	
	
	
	
\end{document}